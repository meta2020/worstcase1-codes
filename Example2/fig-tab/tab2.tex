\begin{table}

\caption{\label{tab:tab2}Example 2: the lower bound of the SAUC by the simulation-based bound.}
\centering
\begin{threeparttable}
\begin{tabular}[t]{rrrr}
\toprule
\multicolumn{1}{c}{} & \multicolumn{1}{c}{Condition (D4.1)} & \multicolumn{1}{c}{Condition (D4.2)} & \multicolumn{1}{c}{Condition (D4.3)} \\
\cmidrule(l{3pt}r{3pt}){2-2} \cmidrule(l{3pt}r{3pt}){3-3} \cmidrule(l{3pt}r{3pt}){4-4}
p & LB (95\% CLB) & LB (95\% CLB) & LB (95\% CLB)\\
\midrule
1.0 & 0.729 (0.645) & 0.729 (0.645) & 0.729 (0.645)\\
0.9 & 0.703 (0.620) & 0.703 (0.620) & 0.703 (0.620)\\
0.8 & 0.682 (0.599) & 0.682 (0.599) & 0.682 (0.599)\\
0.7 & 0.660 (0.578) & 0.660 (0.578) & 0.660 (0.578)\\
0.6 & 0.638 (0.557) & 0.638 (0.557) & 0.638 (0.557)\\
0.5 & 0.613 (0.533) & 0.613 (0.533) & 0.613 (0.533)\\
0.4 & 0.587 (0.507) & 0.587 (0.507) & 0.587 (0.507)\\
0.3 & 0.555 (0.476) & 0.555 (0.476) & 0.555 (0.476)\\
0.2 & 0.514 (0.436) & 0.514 (0.436) & 0.514 (0.436)\\
0.1 & 0.455 (0.379) & 0.455 (0.379) & 0.455 (0.379)\\
\bottomrule
\end{tabular}
\begin{tablenotes}
\item 
           LB indicates lower bound; CLB indicates the confidence lower band of the LB.
\end{tablenotes}
\end{threeparttable}
\end{table}
